\documentclass[margin, line, a4paper]{resume}

\usepackage[utf8]{inputenc} %utf8
\usepackage[english,danish]{babel}
\usepackage[T1]{fontenc}
\usepackage{graphicx,wrapfig}
\usepackage{url}
\usepackage{multicol}
\usepackage[colorlinks=true, a4paper=true, pdfstartview=FitV, linkcolor=blue, citecolor=blue, urlcolor=blue]{hyperref}

\pdfcompresslevel=9

\begin{document}
  
  {\sc \Large Curriculum Vitae -- Héctor Maidana}

  \begin{resume}
    \vspace{0.5cm}
    \begin{wrapfigure}{R}{0.5\textwidth}
      \vspace{-1cm}
      \begin{center}
        \includegraphics[width=0.5\textwidth]{photo}
      \end{center}
    \end{wrapfigure}

    \section{\mysidestyle Información\\Personal}
      Héctor Daniel Maidana \\
      19/03/1989 \\
      28 Años \\
      Aimé Tschiffely 3030 \\ 
      Quilmes Oeste, Buenos Aires \\ 
      Tel: 4250-3962 \\
      Cel: 1565832875 \\
      \href{mailto:hdmaidana@gmail.com}{hdmaidana@gmail.com}
      
      Mi interes por la programación comenzó en mi adolescencia, es por eso que al completar el secundario no dude en dedicarme a dicha disciplina. En el año 2009 inicié la carrera de Programación Informática, de la cual solo me faltan solo TIP. Durante este tiempo desarrollé fuertes conocimientos en los conceptos de programación en general los cuales al día de hoy me ayudan con todos los desafíos laborales y personales que se me presentan.  Poseo una gran capacidad de aprendizaje junto con un fuerte interés y necesidad de incorporación de nuevos conocimientos y experiencias. Me destaco por mi actitud positiva y por una muy buena predisposición para el trabajo en equipo.

    \section{\mysidestyle Habilidades / Conocimientos }

      \begin{multicols}{2}
        \textbf{Programación Orientada a Objetos}
          \begin{itemize}
            \item Patrones de diseño
            \item UML
            \item MVC
            \item Test Unitarios
            \item Test de Integración
          \end{itemize}
          
        \textbf{Programación Web}
          \begin{itemize}
            \item Backend y Frontend
            \item HTML \& CSS
            \item jQuery
            \item Internacionalización y localización
          \end{itemize}

        \textbf{Metodologías de trabajo}
          \begin{itemize}
            \item Scrum
            \item BDD
            \item TDD
            \item Estimación
            \item Refactoring
            \item Integración Continua
            \item Trabajo en equipo
          \end{itemize}

        \textbf{Tecnologías}
          \begin{itemize}
            \item Control de versiones (Git, SVN)
            \item Hibernate
            \item Spring
            \item MySQL
            \item Mongo DB
            \item ZK Framework
            \item Android
            \item Zend Framework
            \item AngularJS
            \item MEAN Stack
          \end{itemize}

        \textbf{Lenguajes de programación}
          \begin{itemize}
            \item Java (Avanzado)
            \item JavaScript(Avanzado)
            \item Python (Intermedio)
            \item PHP(Intermedio)
            \item Ruby (Básico) 
          \end{itemize}

        \textbf{Sistemas Opertaivos}
          \begin{itemize}
            \item Linux distribución Ubuntu (Usuario habitual)
            \item Windows (Usuario habitual)
          \end{itemize}
        \textbf{Otros Softwares}
          \begin{itemize}
            \item Eclipse IDE para Java
            \item AndroidStudio basado en Intellij para Android
            \item Photoshop
            \item Microsoft Office y Openoffice
          \end{itemize}
      \end{multicols}

    \section{\mysidestyle Educación}
      \textbf{E.E.M. N° 6 ``Homero Manzi''  - Florencio Varela (2005-2007)} \\
      Titulo Secundario - Economía y Gestión de las Organizaciones.

      \textbf{Universidad Nacional de Quilmes - Bernal (2009-2016)} \\
      Tecnicatura Universitaria en Programación Informática.

    \section{\mysidestyle Experiencia Laboral}\vspace{1mm}
      \textbf{Full-Stack Developer en Trinomio} \\
      \textit{Junio de 2015 $\rightarrow$ Actualidad} \\
      Al principio llamada Turisweb, es una empresa que se especializaba en desarrollar software enfocado al turismo, teniendo como clientes a empresas importantes como Buquebus. Hoy en día con forme a una nueva sociedad, la empresa Trinomio no solo se dedica a sus productos para turismo sino que también posee una sección de Software Factory donde se desarrollan soluciones para otros dominios. Yo soy parte del equipo de esta última sección, desarrollando diferentes aplicaciones para varios clientes. Esto me da la oportunidad de trabajar con diferentes tecnologías todo el tiempo y aprender cosas nuevas.

      \textbf{Java Developer en Universidad Nacional de Quilmes} \\
      \textit{Abril de 2014 $\rightarrow$ Mayo de 2015 (1 año 2 meses)} \\
      Por un convenio de la Empresa TECSO con la Universidad Nacional de Quilmes se armó un equipo pequeño de desarrolladores para trabajar en un proyecto para la empresa de seguros Río Uruguay Seguros, desempeñando las tareas de soporte del sistema y desarrollo de nuevas funcionalidades.

    \section{\mysidestyle Experiencia Docente}\vspace{1mm}
      \textbf{Auxiliar académico en Universidad Nacional de Quilmes} \\
      \textit{Marzo de 2016 $\rightarrow$ Actualidad} \\
      Formo parte del equipo de la materia Organización y Arquitectura de Computadoras. Como auxiliar académico mi trabajo consiste en dar apoyo a los alumnos en las clases, crear contenido para las prácticas, corregir parciales y explicar algún tema de la parte teórica.

    \section{\mysidestyle Idiomas}
      \begin{description}
        \item[Inglés] Escrito/Oral (Medio), Técnico (Alto)
      \end{description}
  
  \end{resume}
\end{document}